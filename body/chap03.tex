\chapter{两源带延迟最小代价完美匹配的随机近似}


\section{最优解的结构分析} \label{sec:3-optimal}

\begin{figure}[ht]
    \centering
    \begin{minipage}{0.7\textwidth}
        \subfloat[成本函数$odd(t)$和$even(t)$]{
            
\begin{tikzpicture}[x=3pt,y=3pt,yscale=4,xscale=4]
    

\coordinate (t) at (2,0);
\coordinate (tst) at (7,0);

\draw[->] (0,0)--(16,0) node[right, font=\song]{时间};
\draw[->] (0,0)--(0,10) node[above left, font=\song]{成本}; 
\draw[->] (12,0)--(12,10); 

\draw[dash pattern=on 3pt off 3pt on 1pt off 3pt] (0,-1) -- ++(16, 0) node [right] {$a$} (0,-2) -- ++(16, 0) node [right] {$b$}; 
\draw [loosely dashed] (0, 2) node [left]{$1$} -- ++(12, 0) (12, 4) node [left]{$2$}-- (16, 4); 

% even(t)
\draw (0, 0) coordinate (a) 
        -- ([turn]150:5.5) coordinate (b) 
        -- ([turn]-60:2.5) coordinate (c)
        -- ([turn]60:4.5) coordinate (d) 
        -- ([turn]-60:3.5)
        (12, 0) -- (14, 4) -- ++(2,0);

% odd(t)
\draw [color=orange] (0, 2) -- ([turn]-90:2.4596) 
        -- ([turn]60:5.59) 
        -- ([turn]-60:2.012)
        -- ([turn]60:4.8) coordinate (f1)
        -- ([turn]-60:1.6)
        (12, 2) -- ++(4, 0);

\coordinate (f) at (12, 0); 

\foreach \i\p in {1/a,2/b,3/c,4/d, 5/f}{
    \fill (\p |- 0, -1) circle (0.5pt) node [below left] {}
    (\p |- 0, -2) circle (0.5pt) node [below left] {}; 
    \draw (\p |- 0, -1.5) node [anchor=east, xshift=0.2] {$p_{\i}$};
}



\draw [color=green, dashed] (f1|-0, 10) -- (f1|-0,-2); 
\draw (f1|-0,-1.7) node [below]  {$t_u$};
\draw [decorate, decoration = {brace, mirror}] (a|- 0,-2.8) --  (f1|- 0,-2.8) node [midway, below][yshift=-2pt][font=\large] {$f$};

\matrix [draw,below left,outer sep=0pt,inner sep=1.5pt] at (current bounding box.north east) {
  \draw (0,0) -- (1.5,0); & \node [right][font=\small] {$even(t)$}; \\
  \draw [color=orange] (0,0) -- (1.5,0); & \node [right][font=\small]{$odd(t)$};\\
};
\end{tikzpicture}
            \label{fig:costfunction}
        }
    \end{minipage}
    \centering
    \begin{minipage}{0.7\textwidth}
        % \centering
        \subfloat[状态函数$s(t)$]{
            \begin{tikzpicture}[x=3pt,y=3pt,yscale=4,xscale=4, baseline=(current bounding box.north)]
    

\coordinate (t) at (2,0);
\coordinate (tst) at (7,0);

\draw[->] (0,0)--(16.5,0) node[right, font=\song]{时间};
\draw[->] (0,5)--(16.5, 5) node[right, font=\song]{时间};
\draw[->] (0,0)--(0,4.5) node[left, xshift=-5pt, font=\song]{状态值};
\draw[->] (0,5)--(0,9.5) node[left, xshift=-5pt, font=\song]{状态值}; 

\draw[dash pattern=on 3pt off 3pt on 1pt off 3pt] (0,-1) -- ++(16, 0) node [right] {$a$} (0,-2) -- ++(16, 0) node [right] {$b$}; 

% \draw [loosely dashed] (0, 2) node [left]{$1$} -- ++(12, 0) (12, 4) node [left]{$2$}-- (16, 4); 

\draw [loosely dashed] (0, 4) node [left]{$1$} -- ++(16, 0) 
                       (0, 9) node [left]{$1$} -- ++(16, 0);    


\draw (0, 9) coordinate (a) 
        -- ([turn]-135:3.478399)  coordinate (b)
        -- ([turn]90:2.8453)coordinate (c)
        --([turn]-90:2.8453) coordinate (d)
        --([turn]90:3.5) coordinate (f1)
        (12, 9) -- ++(4, -4);

\draw [red, line width=2.25] (0, 5) -- (a) 
        (b) -- (b|-0,9)
        (c) -- (c|-0,5)
        (d) -- (d|-0,9)
        (a|-0,0) -- (a|-0,4)
        (b|-0,0) -- (b|-0,2.4596)
        (c|-0,0) -- (c|-0,3.5524)
        (d|-0,0) -- (d|-0,2.4476)
        (12, 5) -- ++(0, 4)
        (12, 0) -- ++(0, 4);

\draw [fill] (b|-0,2.4596) circle (0.3pt) (c|-0,3.5524) circle (0.3pt) (d|-0,2.4476) circle (0.3pt) ;

\draw [fill=none] (0, 4)  -- ([turn]-135:3.478399) circle (0.4pt)   
        (2.4596, 2.4596) -- ++(0.001, 0) 
        -- ([turn]-45:2.8453) circle (0.4pt)
        (4.4716, 3.5524) -- ++(0.001, 0)
        --([turn]-45:2.8453) circle (0.4pt)
        (6.4716, 2.4476) -- ++(0.001, 0)
        --([turn]-45:3.5) 
        (12, 4) -- ++(4,-4);



\coordinate (e) at (12, 0); 

\foreach \i\p in {1/a,2/b,3/c,4/d,5/e}{
    \fill (\p |- 0, -1) circle (0.5pt) node [below left] {}
        (\p |- 0, -2) circle (0.5pt) node [below left] {}; 
    \draw (\p |- 0, -1.5) node [anchor=east, xshift=0.2] {$p_{\i}$};
}

\draw [color=green, dashed] (f1|-0, 10) -- (f1|-0,-2); 
\draw (f1|-0,-1.7) node [below]  {$t_u$};
\draw [decorate, decoration = {brace, mirror}] (a|- 0,-2.8) --  (f1|- 0,-2.8) node [midway, below][yshift=-2pt][font=\large] {$L_f$};

\end{tikzpicture}    
            \label{fig:statefunction}
        }
    \end{minipage}
    \caption{成本函数$odd(t)$,$even(t)$ 和状态函数$s(t)$}
    \label{fig:Cost_State}
\end{figure}

接下来,对每个独立片段都可以定义其\emph{状态函数}:
\begin{equation*}
    s(t) = 
    \left\{
    \begin{aligned}
        & \frac{odd(t) - even(t) + 1}{2} & \text{若$d$为奇}, \\
        & \frac{even(t) - odd(t) + 1}{2} & \text{若$d$为偶}
    \end{aligned}
    \right.
\end{equation*}
假设片段开始为$0$时刻,$d$是到$t$时刻为止(包括$t$时刻)所到达的请求对数量。在图~\ref{fig:phase_partition}中展示了一个样例,七个请求对的序列被划分为$3$个独立片段。在每个独立片段的每个请求对中,红色线段的长度等于该请求对的状态值。

\begin{figure}[htb]
    \centering
    \usetikzlibrary{decorations.pathmorphing}
\usetikzlibrary{decorations.pathreplacing}

\tikzset{snake it/.style={decorate, decoration=snake}}


\begin{tikzpicture}[x=3pt,y=3pt,yscale=4,xscale=4]
    \def\pairs{{a,b,c,d}}
    \coordinate (t) at (2,0);
    \coordinate (tst) at (7,0);
    
    \draw[->] (0,0)--(25,0) node[right]{时间};
    \draw[->] (0,0) node [anchor=north east] {$0$} --(0,7) node[left] {状态值}; 
    \draw [dash pattern=on 3pt off 3pt on 1pt off 3pt] (0,8) -- (25, 8) node [right] {$b$} (0,9) -- (25, 9) node [right] {$a$}; 
    \draw [loosely dashed] (0, 6) node [left]{$1$} -- (25, 6); 

    \draw (0, 6) coordinate (a) 
                 -- ([turn]-135:6) coordinate (b) 
                 -- ([turn]90:3) coordinate (c) 
                 -- ([turn]-90:2) coordinate (d)
                 -- ([turn]90:5) coordinate (f1);
    \draw  (13, 6) coordinate (e) -- (13, 5.9999) 
                 -- ([turn] 45:2) coordinate (f) 
                 -- ([turn] 90:2) coordinate (f2);  

    \draw  (17, 6) coordinate (g) -- (17, 5.999) 
    -- ([turn] 45:8.4852) coordinate (f3);  

    \foreach \i\p in {1/a,2/b,3/c,4/d,5/e,6/f,7/g}
    {
    \fill (\p |- 0, 8) circle (0.5pt) node [below left] {}
        (\p |- 0, 9) circle (0.5pt) node [below left] {}; 
    \draw (\p |- 0, 9.2) node [anchor=north east, xshift=0.2] {$p_{\i}$};
    }
    \draw ({$(a)!0.5!(b)$} |- 0, 8 ) node [anchor=north] [font=\small] {$l_1$}; 
    \draw ({$(b)!0.5!(c)$} |- 0, 8 ) node [anchor=north] [font=\small] {$l_2$}; 
    \draw ({$(c)!0.5!(d)$} |- 0, 8 ) node [anchor=north] [font=\small] {$l_3$}; 
    \draw ({$(d)!0.5!(e)$} |- 0, 8 ) node [anchor=north] [font=\small] {$l_4$}; 
    \draw ({$(e)!0.5!(f)$} |- 0, 8 ) node [anchor=north] [font=\small] {$l_5$}; 
    \draw ({$(f)!0.5!(g)$} |- 0, 8 ) node [anchor=north] [font=\small] {$l_6$}; 

    \foreach \p in {a,c,e,g}{
        \draw [color=red, line width=2.25] (\p) -- (\p|-0,0); 

    }
    \foreach \p in {b,d,f}{
        \draw [color=red, line width=2.25] (\p) -- (\p|-0,6); 
    }

    \draw ($(a)!0.5!(a|-0,0)$) node [anchor=west]  {$S_1$}; 
    \draw ($(b)!0.5!(b|-0,6)$) node [anchor=west]  {$S_2$}; 
    \draw ($(c)!0.5!(c|-0,0)$) node [anchor=west]  {$S_3$}; 
    \draw ($(d)!0.5!(d|-0,6)$) node [anchor=west]  {$S_4$}; 
    
    \draw [decorate, decoration = {brace}] (a|-0,10) --  (f1|-0,10) node [midway, below][yshift=24pt][font=\large] {$f_1$};
    \draw [decorate, decoration = {brace}] (e|-0,10) --  (f2|-0,10) node [midway, below][yshift=24pt][font=\large] {$f_2$};
    \draw [decorate, decoration = {brace}] (g|-0,10) --  (f3|-0,10) node [midway, below][yshift=24pt][font=\large] {$f_3$};;

\end{tikzpicture}
    \caption{独立片段识别样例}
    \label{fig:phase_partition}
\end{figure}

注意到算法~\ref{alg:PhaseIdentification}中使用请求对的状态值来识别独立片段,下面将状态值和状态函数之间建立联系,即证明$s(t_i) = S_i$ 。


\begin{lemma}\label{state-value}
对任意独立片段$f=(f_1, \cdots, f_k)$和其中的任一请求对$f_1 \leq i \leq f_k$, 有$s(t_i) = S_i$。
\end{lemma}
\begin{proof}
考虑 $i\ge 2$。如果 $i - f_1$ 是奇数,在时间区间 $(t_{i-1},t_i]$ 内,$odd(\cdot)$ 不变,而 $even(\cdot)$ 增加了 $2l_{i-1}$。因此可推出,
\begin{align*}
    s(t_i) & = (even(t_i) - odd(t_i) + 1)/2
     = (even(t_{i-1}) + 2l_{i-1} - odd(t_{i-1}) + 1)/2\\
    & = 1 - (odd(t_{i-1}) - even(t_{i-1}) + 1)/2 + l_{i-1}
     = 1 - s(t_{i-1}) + l_{i-1}.
\end{align*}

对于每个独立片段 $f =(f_1,\cdots,f_k)$,状态函数在时间间隔 $[t_{i-1},t_i]$ 中以单位速率减少。在最后一对 $f_k$ 独立片段之后,令 $t_u$ 为状态函数最早变为 $0$ 的时刻,即 $s(t_u) = 0$。显然,$t_u = t_{f_k} + s(f_k)$。为了完善定义,在 $t_u$ 之后,让 $s(\cdot)$ 保持 $0$,直到下一个独立片段开始。称 $t_u$ 为独立片段 $f$ 的结束时间。由此可以定义几何量 $L_f := t_u - t_{f_1}$ 为独立片段 $f$ 的长度。证毕。
\end{proof}

可以发现,对于每个独立片段$f =(f_1,\cdots,f_k)$,状态函数在时间间隔$[t_{i-1},t_i]$中以单位速率减少。在最后一对$f_k$独立片段之后,令$t_u$为状态函数最早变为$0$的时刻,也就是$s(t_u) = 0$。 显然,$t_u = t_{f_k} + s(f_k)$。为了完善定义,在$t_u$之后,让$s(\cdot)$保持$0$,直到下一个独立片段开始。称$t_u$为独立片段$f$的结束时间。由此可以定义几何量$L_f := t_u - t_{f_1}$为独立片段$f$的长度。

在前文的基础上,定义了一个离线算法\TOPI(独立片段识别算法)。该算法的输入是一个序列,首先使用算法\ref{alg:PhaseIdentification}获得独立片段。对于每个独立片段,如果它是一个独立偶片段,则使用\OPTE 进行处理;否则采用\OPTO 处理。以下定理证明算法\TOPI 是最优的。
\begin{theorem}
\label{thm:algorithmPI}
    独立片段识别算法 \TOPI 为 2-MPMD 问题的最优离线算法。
\end{theorem}

\begin{proof}
考虑一个序列$R_p = (p_1, \cdots, p_n)$。定义一个切片序列$R[i]$为包含前$i$个请求对的子序列,即$R[i] := (p_1, p_2, \cdots, p_i)$。令$R[0]$表示空序列。
对于任意切片序列$R[i]$,如果\TOPI 能在其上产生最优解,则称该切片序列是可验证的。同时,令$TOPI(R[i])$表示\TOPI 在$R[i]$上的解所产生的总成本。接下来归纳法作进行证明。

\textsl{基础情形:}
$TOPI(R[0]) = 0$,且 $TOPI(R[1]) = Cost(\{\Inter(p_1)\})$,
$R[0]$和$R[1]$是可验证的。

\textsl{归纳假设:}假设 $R[i-1]$ 和 $R[i-2]$ 都是可验证的。

\textsl{归纳证明:}
现在验证 $R[i]$。考虑序列 $R[i]$中最后一个请求对$p_i$。令 $f=(f_1,\cdots,f_k)$ 为 \TOPI 算法在解 $R[i]$ 中包含 $p_i$ 的独立片段。由于 $R[i]$ 的最优解必须匹配请求对 $q_j$,因此存在两种可能的匹配情况:
\renewcommand{\labelenumi}{(\arabic{enumi})}
\begin{enumerate}[align=left, labelsep=0.3em, itemindent=2em, leftmargin=0em]
    \item {
        \label{enu:casea}
        $\OPT$ 外部匹配$p_{i-1}$ 和 $p_i$,之后 $\OPT(R[i])=TOPI(R[i-2]) + 2l_{i-1}$。
    }
    \item {
        \label{enu:caseb}
        $\OPT$ 内部匹配$p_i$ ,之后 $\OPT(R[i])=TOPI(R[i-1]) + 1$。
    }
\end{enumerate}

注意算法~\ref{alg:PhaseIdentification}中,当一个新的请求对到达时,有两个不同的选项:
\renewcommand{\labelenumi}{(\alph{enumi})}
\begin{enumerate}[align=left, labelsep=0.3em, itemindent=2em, leftmargin=0em]
    \item {
        \label{enu:decisiona}
        将其添加到当前片段中,然后通过子程序执行$\Exter(p_{i-1}, p_i)$。这对应于情况(\ref{enu:casea}),即$TOPI(R[i]) = TOPI(R[i-2]) + 2l_{i-1}$。
    }\item{
        \label{enu:decisionb}
        开始一个新的片段,然后通过子程序执行$\Inter(p_{i})$。这对应于情况(\ref{enu:caseb}),即$TOPI(R[i]) = TOPI(R[i-1]) + 1$。
    }
\end{enumerate}

考虑在独立片段$f$中的最后一对点$p_i$。不失一般性,假设$i-f_1$是奇数。需要证明\TOPI 对于$p_i$的决策是最优的。
假设对于$p_i$选择了选项(\ref{enu:decisiona})。根据子程序~\ref{alg:PhaseIdentification},条件$l_{i-1} < s(t_{i-1})$成立。 

因此可推出,
\begin{align}
     &  &  l_{i-1} & < s(t_{i-1})  \nonumber\\
     & \Rightarrow & even_f(t_{i-1}) - odd_f(t_{i-1}) & < 1 - 2l_{i-1} \nonumber\\
     & \Rightarrow & even_f(t_{i-2}) - odd_f(t_{i-1}) & < 1 - 2l_{i-1} \label{eq:oddeven}
\end{align}

依据假设$i-f_1$为奇数,因此$R[i-2]$为独立偶片段,$R[i-1]$为独立奇片段。此前定义有\TOPI 在独立片段的成本函数,对应为$even_f(t_{i-2})$和$odd_f(t_{i-1}$,因此可推出:
\begin{align}
    & even_f(t_{i-2}) - odd_f(t_{i-1})=TOPI(R[i-2]) - TOPI(R[i-1])
    \label{eq:topi}
\end{align}

结合不等式\ref{eq:oddeven}和等式\ref{eq:topi},有:
\begin{align}
     TOPI(R[i-2]) + 2l_{i-1} &< TOPI(R[i-1]) + 1 \label{eq:topi2}
\end{align}
即选项(\ref{enu:decisiona})比选项(\ref{enu:decisionb})更好。

因此,算法\TOPI 对每一个请求对始终选择最优匹配,最终产生最优解。证毕。
\end{proof}

注意到从算法\ref{alg:PhaseIdentification}中得到的片段是与成本无关的。全局最优解等于每个独立片段的局部最优解的组合。这也是“独立片段”这个术语的起源。对于单个独立片段,下面的引理表明其长度等于其最优解的成本。这个结论在后续的随机算法竞争分析中非常有用。

\begin{lemma}
\label{lem:Phase_Length}
任一独立片段$f$均满足其长度等于其最优解的成本,即$L_f=Cost_{OPT}(f)$。
\end{lemma}
 
\begin{proof}
假设有一个独立片段 $f = (f_1, f_2, \cdots, f_k)$,其中应用了两个子程序\OPTO 和\OPTE。两个程序分别计算$f$的偶成本函数和奇成本函数,表示为$even_{f}(t)$和$odd_{f}(t)$。注意到 $odd_f(t_{f_1}) = 1$ 和 $even_f(t_{f_1}) = 0$ 。在时间$t_{f_1}$到$t_{f_k}$之间,其中一个函数按照速度$2$增长,而另一个保持不变。

\OPTO 和\OPTE 在$k$为奇数和偶数时分别产生一个$f$的解。如果$k$是奇数,$odd_f(t_{f_k})$等于$Cost_{OPT}(f)$。
因此可推出:
\begin{align*}
& & odd_f(t_{f_k}) + even_f(t_{f_k}) &= 1 + 2*(t_{f_k} - t_{f_1}) \\
& \Rightarrow & s(t_{f_k}) + (t_{f_k} - t_{f_1}) &= odd_f(t_{f_k}) \\
& \Rightarrow & L_f &= odd_f(t_{f_k})
\end{align*}
如果$k$是偶数,可以通过交换上面的$even$和$odd$符号作等价证明。证毕。
\end{proof}

\section{2-竞争的状态化随机延迟算法} \label{sec:3-ralgo}
本节介绍了状态化随机延迟算法(Randomized Delaying State-based Algorithm,\textsc{RDSA}),通过基于状态值的随机延迟决策,该算法可高效处理请求对序列,其竞争比被证明为$2$。同时提出了\ralgo 的广义版本,适用于原始输入序列,并保证了竞争比为$2$。

\subsection{状态化随机延迟算法} \label{subsec:rdsa}

\begin{algorithm}
\caption{状态化随机延迟算法 \ralgo}
\label{alg:randomizedalgo}
\KwIn{在线输入请求对}
\KwOut{匹配解 $M$}
初始化$S_1 \leftarrow 1.$\\
\While{算法空闲时,在时刻$t(p_i)$接收请求对$p_i$}{
    $S_i \leftarrow \min\{1-S_{i-1}+l_{i-1}, 1\}$\\
    采样$X_i \sim \mathbf{U}(0, S_i)$\\
    \eIf{在时间段$( t(p_i), t(p_i)+X_i ]$接收请求对$p_{i+1}$ }{
        $ S_{i+1} \leftarrow \min\{1-S_{i}+l_{i}, 1\}$\\
        $M$ add $\Exter(p_i, p_{i+1})$ at time $t(p_{i+1})$\\
    }{
        $M$ add $\Inter(p_i)$ at time $t(p_i)+X$\\
    }
}
\Return {$M$.}
\end{algorithm}
算法~\ref{alg:PhaseIdentification} 在接收到每个请求对 $p_i$ 时会在线计算其状态值 $S_i$。若\ralgo 在空闲时接收请求对 $p_i$,从 $0$ 到 $S_i$ 中均匀采样值 $X_i$,然后将请求对 $p_i$ 延迟 $X_i$ 秒。如果下一对 $p_{i+1}$ 在不晚于 $t(p_i)+X_i$ 时刻到达,则 \ralgo 会在 $t(p_{i+1})$ 时刻执行 $\Exter(p_i, p_{i+1})$。否则,\ralgo 将在 $t(p_i)+X$ 时刻执行 $\Inter(p_i)$。

由于随机性,无法确定算法\ralgo 在新请求对到达时处于等待或空闲状态。尽管如此,在一个包含 $k$ 个请求对的单个独立片段 $f=(f_1, \cdots, f_k)$ 并且 $f_1\leq i\leq f_k$ 的情况下,将 \ralgo 在请求对 $p_i$ 到达时处于空闲状态的概率,定义为 $P_i$,以下引理证明该概率等于状态值 $s(t(p_i))$。

\begin{lemma} \label{lem:probability}
     对于任意独立片段$f=(f_1, \cdots, f_k)$且满足$f_1\leq i\leq f_k$中, 有$P_i=s(t(p_i))$.
\end{lemma}

\begin{proof}
使用归纳法证明该引理。对第一请求对 $p_{f_1}$,$P_1=s(t(p_{f_1}))=1$。假设 $f_1<i\leq f_k$ 并且 $P_{i-1}=s(t(p_{f_{i-1}}))$。

考虑两个连续的请求对 $p_{i-1}$ 和 $p_i$。当 $p_i$ 到达时,\ralgo 可能处于空闲状态的情况有两种可能,如下所示。
\renewcommand{\labelenumi}{(\arabic{enumi})}
\begin{enumerate}[align=left, labelsep=0.3em, itemindent=2.3em, 
leftmargin=0em]
    \item 当 $p_{i-1}$ 到达时,\ralgo 执行 $\Exter(p_{i-2}, p_{i-1})$ 并等待。这种情况发生的概率为 $1-P_{i-1}$。
    \item 在时间间隔 $[t(p_{i-1}), t(p_i))$ 内,\ralgo 执行 $\Inter(p_{i-1})$,其中 $X_{i-1}$ 的样本小于 $l_{i-1}$。这种情况的概率为 $P_{i-1}\cdot \frac{l_{i-1}}{S_{i-1}} = l_{i-1}$。
\end{enumerate}

总之,当 $p_i$ 到达时,\ralgo 处于空闲状态的概率等于 $P_i = 1 - P_{i-1} + l_{i-1} =s(t(p_i))$。证毕。
\end{proof}

现在,证明随机在线算法\ralgo 在每个独立片段的解上竞争比都为$2$。
\begin{theorem}
    对于任意独立片段$f= (f_1, f_2, \cdots, f_k)$,算法\ralgo 在$f$上的竞争比为$2$。
\end{theorem}

\begin{proof}
设$L_f$为某个独立片段$f$的长度,$t_u$为该独立片段的结束时间。令$rdsa(t)$表示\ralgo 在时间$t$之前所产生的总期望成本(包括空间和时间成本)。根据引理~\ref{lem:Phase_Length},该独立片段上最优解的成本等于$L_f$。使用归纳法验证当$f_1 \leq i \leq f_k$时,有
\begin{align}
rdsa(t(p_i))=2(t(p_i)-t(p_{f_1}))+S_i\cdot (1-S_i) \label{eq:rdsa_pi} 
\end{align}

\textsl{基础情形:}当$i=1$时,$rdsa(t_{f_1})=0$且$S_1=1$,因此直接得到等式~\ref{eq:rdsa_pi}。

\textsl{归纳假设:}当$f_1 \leq i < f_k$,假设等式~\ref{eq:rdsa_pi}对于任意$i$成立,现在证明其对于$i+1$也成立。

\textsl{归纳证明:}考虑时间区间$(t(p_i),t(p_{i+1})]$。相较于$rdsa(t(p_i))$,如果\ralgo 在$p_i$到达时处于空闲状态,则$rdsa(t(p_{i+1}))$会包含一些额外的成本,这种情况的概率是$P_i$。

现在讨论\ralgo 为$p_i$选择的延迟时间$X_i$。如果$X_i<l_i$,则\ralgo 会在时间$t(p_i)+X_i$执行$\Inter(p_i)$,额外成本为$2X_i+1$。否则,$X_i \geq l_i$,并且\ralgo 会执行$\Exter(p_i,p_{i+1})$,额外成本为$2l_i$。总之,时间区间$(t(p_i),t(p_{i+1})]$内产生的额外成本期望为
$$
    P_i \cdot \big( \int_{0}^{l_i} \frac{2X_i+1}{S_i}d(X_i) + \int_{l_i}^{S_i} \frac{2l_i}{S_i}d(X_i)\big) = l_i\cdot (1 + 2S_i - l_i)
$$
因此,
\begin{align*}
    rdsa(t(p_{i+1})) &= rdsa(t(p_{i})) + l_i\cdot (1 + 2S_i - l_i) \\
    &= 2(t(p_i)-t(p_{f_1}))+S_i\cdot (1-S_i) + l_i\cdot (1 + 2S_i - l_i)\\
    &=2(t(p_{i+1}) - t(p_{f_1})) + S_{i+1}\cdot (1-S_{i+1})
\end{align*}

由上可知,等式~\ref{eq:rdsa_pi}对于任何$f_1 \leq i \leq f_k$都成立。

最后计算$(t(p_{f_k}), t_u]$的期间成本,其中只包含$t(p_{f_k})+X_{f_k}$时刻的$\Inter(p_{f_k})$的成本为$P_{f_k}\cdot (2\mathbf{E}[X_{f_k}]+1) = S_{f_k}\cdot (S_{f_k}+1)$。
因此,$rdsa(t_u) = rdsa(t(p_{f_k})) + S_{f_k}\cdot (S_{f_k}+1) = 2L_f$。证毕。
\end{proof}



在定理~\ref{thm:algorithmPI}中证明了最优解的成本是所有独立片段成本之和,因此每个独立片段对于最优算法来说是独立的。同样地,在状态化随机延迟算法中,由于该算法是基于状态值$S_i$选择选项,每个独立片段都以初始状态值$1$开始。因此,上一个独立片段的解不会影响当前独立片段的解的选择。对于\ralgo 来说,所有独立片段也是独立的。

综上所述,状态化随机延迟算法的竞争比为
\begin{equation} \label{eq:cost_ratio}
\frac{Cost_{RDSA}(R_p)}{Cost_{OPT}(R_p)} = \frac{\sum_{f}Cost_{RDSA}(f)}{\sum_{f}Cost_{OPT}(f)} =
\frac{2\sum_{f}Cost_{OPT}(f)}{\sum_{f}Cost_{OPT}(f)} = 2    
\end{equation}

\begin{corollary}
    \label{cor:2-competitive}
    对任意请求对序列$R$,状态化随机延迟算法\ralgo 在$R$上的竞争比为$2$。
\end{corollary}

\subsection{广义版本的状态化随机延迟算法\ralgo}\label{subsec:generalization}
本节介绍了\ralgo 的广义版本,适用于原始序列。假设给定请求序列$R$由$n$个请求组成,$r_1, \ldots, r_n$,其中 $t(r_1) \leq \ldots \leq t(r_n)$。对于任何有效算法$ALG$,每次最多有两个开放的请求。通过这种方式,可以根据开放的请求数量的奇偶性将时间轴分为“奇数区”和“偶数区”。更准确地说,对于$1\leq i \leq n$,当$i$为奇数时,区间$[t(r_i), t(r_{i+1}))$被称为奇数区;当$i$为偶数时,该区间被称为偶数区\footnote{此处定义$t(r_{n+1})=+\infty.$}。请注意,对于每个有效算法,包括最优算法,在每个奇数区间内都只有一个开放的请求,这导致不可避免的等待成本。在广义版本中仍然将某些请求成对考虑,并为每对请求定义到达时间。接下来按以下方式呈现\ralgo 的广义版本。

开始时,没有请求,算法处于空闲状态。

当算法处于空闲状态时,检查接下来到达的两个请求 $r_i$ 和 $r_{i+1}$(其中 $i$ 为奇数)。时间区间 $[t(r_i), t(r_{i+1}))$ 是奇数区间。在时间 $t(r_{i+1})$,如果 $x(r_i)=x(r_{i+1})$,则算法匹配这两个请求并保持空闲状态。否则,算法会将这两个请求视为到达时间为 $t(r_{i+1})$ 的一对请求,并进入等待状态。该对请求也被分配了一个类似于算法~\ref{alg:PhaseIdentification} 中的状态值 $S$。如果它是第一对,则其状态值为 $1$。否则,它的状态值等于 $\min\{1-S'+l,1\}$,其中 $S'$ 是前一对的状态值,$l$ 是从上一对到 $t(r_{i+1})$ 到达时间之间偶数区间的总长度。同样从 $\mathbf{U}(0,S)$ 中抽样一个延迟时间 $X$。

当算法处于等待状态时,设 $R_1$ 和 $R_2$ 是具有状态值 $S$ 的一对开放请求,其中 $x(R_1)=a$,$x(R_2)=b$。类似地,检查接下来到达的两个请求 $r_i,r_{i+1}$,并且时间区间 $[t(r_i),t(r_{i+1}))$ 是奇数区间。不失一般性,假设 $x(r_i)=a$。在时间 $t_i$,算法匹配请求 $R_1$ 和 $r_i$。如果 $x(r_{i+1})=a$,则请求 $R_2$ 和 $r_{i+1}$ 被视为一对具有相同状态值和到达时间的请求。否则,当 $x(r_{i+1})=b$ 时,算法匹配 $R_2$ 和 $r_{i+1}$ 并变为空闲状态。同时,请求 $r_i$ 和 $r_{i+1}$ 被视为一对,其状态值为 $\min\{1-S'+l,1\}$,而到达时间为 $t(r_{i+1})$,这与之前类似。请注意,算法不会等待太久。更确切地说,一旦从时间 $\min\{t(R_1),t(R_2)\}$ 开始,偶数区间的总长度达到 $X$,则请求 $R_1$ 和 $R_2$ 立即匹配,并且算法变为空闲状态。

以下定理显示 \ralgo 的广义版本也具有竞争比 $2$。

\begin{theorem}
算法 \ralgo 的广义版本对于在线 2-MPMD 问题是 $2$ 竞争比的。
\end{theorem}

\begin{proof}
    考虑一个奇数区间$[t(p_i),t(p_{i+1}))$,其中$i$是奇数。设$\Delta t=t(p_{i+1})-t(p_i)$。对于任何有效算法,包括 \ralgo 的广义版本,在此时间间隔内恰好存在一个开放的请求。对于所有不早于$t(p_{i+1})$到达的请求,让它们提前$\Delta t$到达。这样,最优解的成本减少了$\Delta t$。与此同时,\ralgo 广义版本产生的解的成本也减少了$\Delta t$,即其竞争比不会增加。
    
    现在,任何奇数区间的长度都为零,即输入序列已被改变为成对序列。广义版的 \ralgo 与原始版本相同。根据推论\ref{cor:2-competitive},该算法是$2$-竞争的。证毕。
\end{proof}


\section{随机近似下界的归纳证明} \label{sec:3-lowerbound}
本节采用姚氏最小最大原则 (Yao's principle) \cite{yao1977probabilistic} 推导 2-MPMD 问题的随机近似下界,并表明状态化随机延迟算法是理论最优的。通过构建一种随机序列的分布,使得任何在线确定性算法的期望成本比至少为$2$。

\subsection{随机序列构建}
构造过程中使用状态函数$s(t)$来指示构造\footnote{这仍然是一个状态化过程,构造的当前步骤会基于上一个步骤的状态},注意到状态函数$s(t)$可以由折线描述,其中每个转折点对应于序列中的一个请求对,如图~\ref{fig:statefunction}所示。随机匹配序列 $\tlr$ 构造过程如下:

\renewcommand{\labelenumi}{(\arabic{enumi})}
\begin{enumerate}[align=left, labelsep=-0.5em, itemindent=3.2em, leftmargin=0em]
\item 将第一对放置在时间 $0$,并将其状态值设置为 $1$。
\item 对于上一次放置的请求对,假设其拥有到达时间 $t$ 和状态值 $S=s(t)$,采样一个非负时间间隔$X\sim \mathrm{Exp}(1)$\footnote{回想一下 $\mathrm{Exp}(1)$ 是指数分布,其概率密度函数为 $f(x)=e^{-x}$,其中 $x \geq 0$。}。
\item 如果 $X<S$,则在时间 $t+X$ 放置一请求对,并将其状态值设为 $1-S+X$;否则终止过程。
\item 重复步骤2和3,直到过程结束。
\end{enumerate}


由此得到的序列实际上是一个独立片段。假设存在一个请求对到达时间为$t$、状态值为$S=s(t)$,令$P_t(x)$ 是在间隔 $(t,t+x)$ 内未进行任何放置的概率。因此,当 $x<s(t)$ 时,$P_t(x)=\int_{0}^{x}e^{-y}dy=e^{-x}$,与 $t$ 无关。为简单起见,在 $x\leq s(t)$ 时使用 $P(x)$ 代替记号 $P_t(x)$,且 $P(x)=e^{-x}$。

\subsection{竞争比分析}




现在能够计算$OPT$和任何在线确定性算法$ALG$产生的期望成本。需要注意的是,构造过程是状态化的,也就是对状态值为$S$的任意时间点后构造的序列的分布只与$S$有关。首先讨论$OPT$产生的期望成本,由引理~\ref{lem:Phase_Length}可知这等于相应独立片段长度的期望值。

\begin{lemma}\label{lemma:optcost}
    最优算法在$\tlr$的期望成本至多为$1$。
\end{lemma}
    
\begin{proof}
对于 $1\leq i$ 且 $0\leq s\leq 1$,假设第 $i$ 请求对在时间 $t$ 以状态值 $s$ 放置,则 $T(i,s)$ 表示第 $i$ 对之后的独立片段期望长度。注意到 $T(1,1)$ 刚好是所希望的。第$(i+1)$ 请求对有以下两种可能情况:
\renewcommand{\labelenumi}{(\arabic{enumi})}
\begin{enumerate}[align=left, labelsep=0.5em, itemindent=2.5em, leftmargin=0em]
    \item 第$(i+1)$ 请求对不存在,即序列的结束时间为 $t+s$。该情况的概率为 $P(s)=e^{-s}$,额外成本为 $s$,如图~\ref{fig:opt_1}所示。
    \item 第$(i+1)$ 请求对的到达时间为 $t+x$,其中 $0\leq x < s$,其概率密度为 $P(x)=e^{-x}$,额外成本为 $x+T(i+1,1-s+x)$,如图~\ref{fig:opt_2}所示。
\end{enumerate}

因此,对于$1\leq i$和$0\leq s \leq 1$,
$$
    T(i,s)=s\cdot e^{-s}+\int_{0}^{s}(x+T(i+1,1-s+x))\cdot e^{-x}dx
$$
假设$L>0$是足够大的常数,定义$T'(i,s)$为辅助函数,使得当$i\geq L$时,$T'(i,s)$满足与$T'(i,s)=-s^2+2s$相同的方程。那么,对于$i=L-1,\cdots,1$,使用归纳法如下:
\begin{align*}
    T'(i,s)&=s\cdot e^{-s}+\int_{0}^{s}(x+T'(i+1,1-s+x))\cdot e^{-x}dx \\
    &=s\cdot e^{-s}+\int_{0}^{s}(x-(1-s+x)^2+(1-s+x))\cdot e^{-x}dx \ \ \text{(归纳假设)}\\
    &=-s^2+2s
\end{align*}
因此,对于所有$i \geq 1$和$0 \leq s \leq 1$,有$T'(i,s)=-s^2+2s$。具体来说,$T'(1,1)=1$。现在考虑$T(1,1)$和$T'(1,1)$之间的差距。对于$\ell \geq L$,独立片段包含至少$\ell$对的概率最多为$(1-P(1))^{\ell}=(1-e^{-1})^{\ell}$。对于具有$\ell$对且第$L$对状态值为$s$的独立片段,在第$L$对之后实际发生的成本最多为$\ell-L+1$\footnote{简单地内部匹配所有对,这不会引起比最优算法更少的成本},其中在$T'$中将其视为$-s^2+2s\geq 0$。因此,
\begin{align*}
    T(1,1)\leq T'(1,1)+\sum_{\ell=L}^{+\infty} (\ell-L+1)\cdot (1-e^{-1})^{\ell} = 1+e^2\cdot (1-e^{-1})^L.
\end{align*}
当$L$可以任意大时,$T(1,1)\leq 1$。证毕。 
\end{proof}
\begin{figure}
  \centering
  \begin{tabular}{cc}
% 第一行

\begin{minipage}[b]{0.5\textwidth} \subfloat[$OPT$情况1]{
\label{fig:opt_1}
\centering
\begin{tikzpicture}[scale=0.5]
\tikzset{
  pnode/.style={
    circle, 
    draw,
    fill=black,
    inner sep=1.5pt,
    outer sep=0pt
  }
}
\draw[->] (0,0) -- (11,0);
\draw[->] (0,0) -- (0,9) node[above left] {状态值};
\draw (9.5, 0)node[below right] {时间};
\draw[red, line width=2pt] (2,0) -- (2,6) node[black, right] {};
\draw[dashed] (0,8) -- (11,8);
\draw (2,0) node[below] {$t$};
% \draw (2,2.5) node[left] {$s(t)$};
\filldraw [black] (0,8) circle[radius=2pt];
\draw (0,8) node[left] {$1$};
\draw[dashed] (0, 6)node [left] {$s$} -- +(2,0) ;
\draw[dash pattern=on 3pt off 3pt on 1pt off 3pt] (0,10) -- (10,10) node[right] {$a$};
\draw[dash pattern=on 3pt off 3pt on 1pt off 3pt] (0,12) -- (10,12) node[right] {$b$};

\draw (2,6) -- (8, 0) node[below] {$t+s$};

\coordinate (i) at (2,6);

\draw (i|- 0,12) node[pnode]  {};
\draw (i|- 0,10) node[pnode]  {};
\draw (i|- 0,11) node {\large$p_i$};

\end{tikzpicture}}
\end{minipage}\hfil &

\begin{minipage}[b]{0.5\textwidth} \subfloat[$OPT$情况2]{
\label{fig:opt_2}
\centering
\begin{tikzpicture}[scale=0.5]
\tikzset{
  pnode/.style={
    circle, 
    draw,
    fill=black,
    inner sep=1.5pt,
    outer sep=0pt
  }
}
\draw[->] (0,0) -- (11,0);
\draw[->] (0,0) -- (0,9) node[above left] {状态值};
\draw (9.5, 0)node[below right] {时间};
\draw[red, line width=2pt] (2,0) -- (2,6) node[black, right] {};
\draw[dashed] (0,8) -- (11,8);
\draw (2,0) node[below] {$t$};
% \draw (2,2.5) node[left] {$s(t)$};
\filldraw [black] (0,8) circle[radius=2pt];
\draw (0,8) node[left] {$1$};
\draw[dashed] (0, 6)node [left] {$s$} -- +(2,0) ;
\draw[dash pattern=on 3pt off 3pt on 1pt off 3pt] (0,10) -- (10,10) node[right] {$a$};
\draw[dash pattern=on 3pt off 3pt on 1pt off 3pt] (0,12) -- (10,12) node[right] {$b$};

\coordinate (p) at (6,2);
\coordinate (i) at (2,6);

\draw (i) -- (p);
\draw[dashed] (p) -- (6,0) node[below] {$t+x$};

\draw[red,line width=2pt] (p) -- (p|-0,8) node[black, right] {};
\draw (p) -- ++(45:0.3cm) sin ++(45:0.3cm) cos ++(45:0.3cm) sin ++(45:0.3cm) cos ++(45:0.3cm) sin ++(45:0.3cm) cos ++(45:0.3cm) sin ++(45:0.3cm) cos ++(45:0.3cm);

\draw (i|- 0,12) node[pnode]  {} (i|- 0,10) node[pnode]  {} (p |- 0,12) node[pnode]  {} (p |- 0,10) node[pnode] {};
\draw (i|- 0,11) node {\large$p_i$} (p|-0,11) node {\large$p_{i+1}$};

\end{tikzpicture}}
\end{minipage} \\

\end{tabular}
  \caption{最优算法$OPT$在$\tlr$上的期望成本}
\end{figure}

现在讨论任何在线确定性算法产生的期望成本。

\begin{lemma} \label{lemma:alg_cost}
对于任何在线确定性算法 $ALG$,$ALG$ 所产生的期望成本至少为 $2$。    
\end{lemma}
\begin{proof}
假设对于$1\leq i$和$0\leq s\leq 1$,第$i$对数$p_i$在时间$t$以状态值$s$放置。令$ALG(i,s,0)$和$ALG(i,s,1)$分别表示$ALG$在接收$p_i$后处于空闲或等待状态时产生的期望成本。目标是$ALG(1,1,1)\ge 2$。

假设当接收到配对$p_i$时,$ALG$处于空闲状态。如果$p_i$是该独立片段的最后一对,则不会产生额外成本。否则,假设在$p_i$之后的第一对在时间$t+x$放置,其概率密度为$e^{-x}$。按照这种方式,$ALG$在时间$t+x$开始等待,这表明
$$ALG(i,s,0)=\int_{0}^{s}ALG(i+1,1-s+x,1)\cdot e^{-x}dx$$ 该情况如图~\ref{fig:alg-4}所示。

现在考虑当 $ALG$ 收到对 $(p_i)$ 时处于等待状态的情况。假设 $ALG$ 将对 $p_i$ 延迟 $m$ 秒,其中 $m\leq s$。有以下三种可能情况。

\renewcommand{\labelenumi}{(\arabic{enumi})}
\begin{enumerate}[align=left, labelsep=0.5em, itemindent=2.5em, leftmargin=0em]
    \item 第 $(i+1)$ 对不存在,即序列的结束时间是 $t+s$。此情况的概率为 $P(s)=e^{-s}$,$ALG$ 在时间 $t+m$ 执行 $\Inter(p_i)$,额外成本为 $2m+1$,如图~\ref{fig:alg-1}所示。
    \item 第 $(i+1)$ 对 $p_{i+1}$ 的到达时间为 $t+x$,其中 $x\leq m$,其概率密度为 $e^{-x}$。$ALG$ 在时间 $t+x$ 执行 $\Exter(p_i,p_{i+1})$ 并转为空闲状态,额外成本为 $2x+ALG(i+1,1-s+x,0)$如图~\ref{fig:alg-2}所示。
    \item 第 $(i+1)$ 对 $p_{i+1}$ 的到达时间为 $t+x$,其中 $x>m$,其概率密度为 $e^{-x}$。$ALG$ 在时间 $t+m$ 执行 $\Inter(p_i)$ 并转为空闲状态。在时间 $t+x$,它再次转为等待状态。此情况下的额外成本为 $2m+1+ALG(i+1,1-s+x,1)$如图~\ref{fig:alg-3}所示。
\end{enumerate}

因此,对于$1\leq i$且$0\leq s\leq 1$,有:
\begin{align*}
ALG(i,s,1)=(2m+1)\cdot e^{-s} & +\int_{0}^{m}(2x+ALG(i+1,1-s+x,0))\cdot e^{-x}dx \\
& + \int_{m}^{s}(2m+1+ALG(i+1,1-s+x,1))\cdot e^{-x}dx
\end{align*}


与引理~\ref{lemma:optcost}类似,令$L>0$为足够大的常数,并定义$ALG'(i,s,0)$和$ALG'(i,s,1)$作为两个辅助函数,使得相同的方程成立且在$i\geq L$和$j\in \{0,1\}$时,$ALG'(i,s,j)=j-s^2+2s$。然后,对于$i=L-1,\cdots,1$,使用归纳方法如下:
\begin{align*}
    ALG'(i,s,0)&=\int_{0}^{s}ALG'(i+1,1-s+x,1)\cdot e^{-x}dx \\
    &=\int_{0}^{s}(1-(1-s+x)^2+2(1-s+x))\cdot e^{-x}dx \\
    &=-s^2+2s \\
    ALG'(i,s,1)&=(2m+1)\cdot e^{-s}+\int_{0}^{m}(2x+ALG'(i+1,1-s+x,0))\cdot e^{-x}dx \\
    &+\int_{m}^{s}(2m+1+ALG'(i+1,1-s+x,1)\cdot e^{-x}dx \\
    &=(2m+1)\cdot e^{-s}+\int_{0}^{m}(2x-(1-s+x)^2+2(1-s+x))\cdot e^{-x}dx \\
    &+\int_{m}^{s}(2m+2-(1-s+x)^2+2(1-s+x))\cdot e^{-x}dx \\
    &=1-s^2+2s
\end{align*}

与引理~\ref{lemma:optcost}类似,设$L>0$为一个足够大的常数,并定义$ALG'(i,s,0)$和$ALG'(i,s,1)$为两个辅助函数,使得相同的方程成立,当$i\ge L$且$j\in\{0,1\}$时,有$ALG'(i,s,j)=j-s^2+2s$。然后,对于$i=L-1,\cdots,1$,使用归纳法如下:
\begin{align*}
    ALG'(i,s,0)&=\int_{0}^{s}ALG'(i+1,1-s+x,1)\cdot e^{-x}dx \\
    &=\int_{0}^{s}(1-(1-s+x)^2+2(1-s+x))\cdot e^{-x}dx \\
    &=-s^2+2s \\
    ALG'(i,s,1)&=(2m+1)\cdot e^{-s}+\int_{0}^{m}(2x+ALG'(i+1,1-s+x,0))\cdot e^{-x}dx \\
    &+\int_{m}^{s}(2m+1+ALG'(i+1,1-s+x,1)\cdot e^{-x}dx \\
    &=(2m+1)\cdot e^{-s}+\int_{0}^{m}(2x-(1-s+x)^2+2(1-s+x))\cdot e^{-x}dx \\
    &+\int_{m}^{s}(2m+2-(1-s+x)^2+2(1-s+x))\cdot e^{-x}dx \\
    &=1-s^2+2s
\end{align*}



因此,当$i \geq L$且$j \in \{0,1\}$时,$ALG'(i,s,j)=j-s^2+2s$。特别地,$ALG'(1,1,1)=2$。现在,让考虑$ALG(1,1,1)$和$ALG'(1,1,1)$之间的差距。对于$\ell \geq L$,独立片段包含至少$\ell$对的概率最多为$(1-P(1))^{\ell}=(1-e^{-1})^{\ell}$。对于一个具有$\ell$个请求对,并且第$L$对具有状态值$s$的独立片段,在第$L$对之后实际产生的成本不小于$0$,其中在$T'$中被认为是至多$1-s^2+2s\geq 2$。因此,
\begin{align*}
    ALG(1,1,1)\geq ALG'(1,1,1)-\sum_{\ell=L}^{+\infty}2(1-e^{-1})^{\ell}=2-2e(1-e^{-1})^L
\end{align*}
由于$L$可以任意大,$ALG(1,1,1)\geq 2$。证毕。
\end{proof}

\begin{figure}[htp]
  \centering
  \begin{tabular}{cc}
% 第一行

\begin{minipage}[b]{0.5\textwidth} \subfloat[$ALG$在$t$处于等待时情况1]{
\label{fig:alg-1}
\centering
\begin{tikzpicture}[scale=0.5]

\tikzset{
  pnode/.style={
    circle, 
    draw,
    fill=black,
    inner sep=1.5pt,
    outer sep=0pt
  }
}
\draw[->] (0,0) -- (11,0);
\draw[->] (0,0) -- (0,9) node[above left] {状态值};
\draw (9.5, 0)node[below right] {时间};
\draw[red, line width=2pt] (2,0) -- (2,6) node[black, right] {};
\draw[dashed] (0,8) -- (11,8);
\draw (2,0) node[below] {$t$};
% \draw (2,2.5) node[left] {$s(t)$};
\filldraw [black] (0,8) circle[radius=2pt];
\draw (0,8) node[left] {$1$};
\draw[dashed] (0, 6)node [left] {$s$} -- +(2,0) ;
\draw[dash pattern=on 3pt off 3pt on 1pt off 3pt] (0,10) -- (10,10) node[right] {$a$};
\draw[dash pattern=on 3pt off 3pt on 1pt off 3pt] (0,12) -- (10,12) node[right] {$b$};
\coordinate (p) at (5,3);
\coordinate (i) at (2,6);

\draw (i) -- (8,0);
% \draw[red, line width=2pt] (p) -- (p|-0,8) node[black, right] {};


\draw (i|- 0,12) coordinate (i1) node[circle, fill, inner sep=1.5pt]{};
\draw (i|- 0,11) node {\large$p_i$};
\draw (i|- 0,10) coordinate (i2)  node[circle, fill, inner sep=1.5pt]{};

\draw[green, line width=1pt] (i1) -- (i1-|p) coordinate(p1) -- (p|-0,10) coordinate(p2) -- (i2);
\draw (p1)  node [above] {$\texttt{Internal}(p_i)$};
\draw[dashed] (p2) -- (p2|-0,0) node [below] {$t+m$};

\end{tikzpicture}}
\end{minipage}\hfil &

\begin{minipage}[b]{0.5\textwidth} \subfloat[$ALG$在$t$处于等待时情况2]{
\label{fig:alg-2}
\centering
\begin{tikzpicture}[scale=0.5]

\tikzset{
  pnode/.style={
    circle, 
    draw,
    fill=black,
    inner sep=1.5pt,
    outer sep=0pt
  }
}
\draw[->] (0,0) -- (11,0);
\draw[->] (0,0) -- (0,9) node[above left] {状态值};
\draw (9.5, 0)node[below right] {时间};
\draw[red, line width=2pt] (2,0) -- (2,6) node[black, right] {};
\draw[dashed] (0,8) -- (11,8);
\draw (2,0) node[below] {$t$};
% \draw (2,2.5) node[left] {$s(t)$};
\filldraw [black] (0,8) circle[radius=2pt];
\draw (0,8) node[left] {$1$};
\draw[dashed] (0, 6)node [left] {$s$} -- +(2,0) ;
\draw[dash pattern=on 3pt off 3pt on 1pt off 3pt] (0,10) -- (10,10) node[right] {$a$};
\draw[dash pattern=on 3pt off 3pt on 1pt off 3pt] (0,12) -- (10,12) node[right] {$b$};
\coordinate (p) at (4,4);
\coordinate (m) at (5,0);
\coordinate (i) at (2,6);

\draw (i) -- (p);
\draw[red, line width=2pt] (p) -- (p|-0,8) node[black, right] {};

\draw (i|- 0,12) coordinate (i1) node[circle, fill, inner sep=1.5pt]{};
\draw (i|- 0,10) coordinate (i2)  node[circle, fill, inner sep=1.5pt]{};
\draw (i|- 0,11) node {\large$p_i$};

\draw[green, line width=1pt] (i1) -- (i1-|p) coordinate(p1);
\draw[green, line width=1pt] (p|-0,10) coordinate(p2) -- (i2);

\draw (p1)  node [above] {$\texttt{External}(p_i,p_{i+1})$};
\draw[dashed] (p2) -- (p2|-0,0) -- ++(-0.5,0) node [below] {\small$t+x$};

\draw (p1) node[circle, fill, inner sep=1.5pt]{};
\draw (p2) node[circle, fill, inner sep=1.5pt]{};
\draw (p|- 0,11) node {\large$p_{i+1}$};

\draw (p) -- ++(45:0.3cm) sin ++(45:0.3cm) cos ++(45:0.3cm) sin ++(45:0.3cm) cos ++(45:0.3cm) sin ++(45:0.3cm) cos ++(45:0.3cm) sin ++(45:0.3cm) cos ++(45:0.3cm);

\draw[dashed] (m) -- (m|-0,10);
\draw (5.5,0) node [below] {\small$t+m$};

\end{tikzpicture}}
\end{minipage}\\


% 第二行
\begin{minipage}[b]{0.5\textwidth} \subfloat[$ALG$在$t$处于等待时情况3]{
\label{fig:alg-3}
\centering
\begin{tikzpicture}[scale=0.5]

\tikzset{
  pnode/.style={
    circle, 
    draw,
    fill=black,
    inner sep=1.5pt,
    outer sep=0pt
  }
}
\draw[->] (0,0) -- (11,0);
\draw[->] (0,0) -- (0,9) node[above left] {状态值};
\draw (9.5, 0)node[below right] {时间};
\draw[red, line width=2pt] (2,0) -- (2,6) node[black, right] {};
\draw[dashed] (0,8) -- (11,8);
\draw (2,0) node[below] {$t$};
% \draw (2,2.5) node[left] {$s(t)$};
\filldraw [black] (0,8) circle[radius=2pt];
\draw (0,8) node[left] {$1$};
\draw[dashed] (0, 6)node [left] {$s$} -- +(2,0) ;
\draw[dash pattern=on 3pt off 3pt on 1pt off 3pt] (0,10) -- (10,10) node[right] {$a$};
\draw[dash pattern=on 3pt off 3pt on 1pt off 3pt] (0,12) -- (10,12) node[right] {$b$};
\coordinate (p) at (6,2);
\coordinate (m) at (5,0);
\coordinate (i) at (2,6);

\draw (i) -- (p);
\draw[red, line width=2pt] (p) -- (p|-0,8) node[black, right] {};

\draw (i|- 0,12) coordinate (i1) node[circle, fill, inner sep=1.5pt]{};
\draw (i|- 0,10) coordinate (i2)  node[circle, fill, inner sep=1.5pt]{};
\draw (i|- 0,11) node {\large$p_i$};

\draw[green, line width=1pt] (i1) -- (i1-|m) -- (m|-0,10) -- (i2);
\draw (m|-0,12)  node [above] {$\texttt{Internal}(p_i)$};


\draw (p|-0,10) node[circle, fill, inner sep=1.5pt]{};
\draw (p|-0,12) node[circle, fill, inner sep=1.5pt]{};
\draw (p|- 0,11) node {\large$p_{i+1}$};
\draw[dashed] (p|-0,10) -- (p|-0,0) -- ++(0.5,0) node [below] {\small$t+x$};

\draw (p) -- ++(45:0.3cm) sin ++(45:0.3cm) cos ++(45:0.3cm) sin ++(45:0.3cm) cos ++(45:0.3cm) sin ++(45:0.3cm) cos ++(45:0.3cm) sin ++(45:0.3cm) cos ++(45:0.3cm);

\draw[dashed] (m) -- (m|-0,10);
\draw (4.5,0) node [below] {\small$t+m$};
\end{tikzpicture}}
\end{minipage}\hfil &


\begin{minipage}[b]{0.5\textwidth} \subfloat[$ALG$在$t$处于空闲状态]{
\label{fig:alg-4}
\centering
\begin{tikzpicture}[scale=0.5]
\tikzset{
  pnode/.style={
    circle, 
    draw,
    fill=black,
    inner sep=1.5pt,
    outer sep=0pt
  }
}
\draw[->] (0,0) -- (11,0);
\draw[->] (0,0) -- (0,9) node[above left] {状态值};
\draw (9.5, 0)node[below right] {时间};
\draw[red, line width=2pt] (2,0) -- (2,6) node[black, right] {};
\draw[dashed] (0,8) -- (11,8);
\draw (2,0) node[below] {$t$};
% \draw (2,2.5) node[left] {$s(t)$};
\filldraw [black] (0,8) circle[radius=2pt];
\draw (0,8) node[left] {$1$};
\draw[dashed] (0, 6)node [left] {$s$} -- +(2,0) ;
\draw[dash pattern=on 3pt off 3pt on 1pt off 3pt] (0,10) -- (10,10) node[right] {$a$};
\draw[dash pattern=on 3pt off 3pt on 1pt off 3pt] (0,12) -- (10,12) node[right] {$b$};

\coordinate (p) at (6,2);
\draw (2,6) -- (p);
\draw[dashed] (p) -- (6,0) node[below] {$t+x$};

\draw[red,line width=2pt] (p) -- (p|-0,8) node[black, right] {};
\draw (p) -- ++(45:0.3cm) sin ++(45:0.3cm) cos ++(45:0.3cm) sin ++(45:0.3cm) cos ++(45:0.3cm) sin ++(45:0.3cm) cos ++(45:0.3cm) sin ++(45:0.3cm) cos ++(45:0.3cm);

\draw (p |- 0,12)  node[circle, fill, inner sep=1.5pt] {};
\draw (p |- 0,10) node[circle, fill, inner sep=1.5pt] {};

\end{tikzpicture}}
\end{minipage}\\
\end{tabular}
  \caption{确定性算法$ALG$在$\tlr$上的期望成本}
\end{figure}

\begin{theorem}
    任何随机化在线算法在2-MPMD上的竞争比至少为 $2$。
\end{theorem}

\begin{proof}
    依据引理~\ref{lemma:optcost}和引理~\ref{lemma:alg_cost},$ALG(1,1,1)\geq2$且$T(1,1)\leq 1$,即对于 $\tlr$,任何在线确定性算法的期望成本比率至少为 $2$。基于姚氏最小最大原则\cite{yao1977probabilistic},任何随机化在线算法的竞争比至少为 $2$。证毕。
\end{proof}

\section{随机近似下界的演绎推导} \label{sec:3-formalization}
本节将介绍解决随机理论下界中使用到的技术。

通过直觉可知需要构造具有某种分布的独立片段,并且这个过程是一个状态化的过程。但是这个分布是未知的。由第~\ref{sec:3-optimal}节可知,最优解的产生完全基于状态值,因此可根据生成片段时当前的状态值来决定之后的片段如何生成。即在某一个时刻,假设该时刻的状态值为$s$,令其之后$\Delta t$时间内生成一个请求对的概率应该为$f(s)\Delta t$,也就是构造序列的生成概率密度为$f(s)$。

接下来计算一个重要的条件概率$g_s(y)$,该条件概率指示在一个状态值为$s$的时间点之后,一直持续到状态值为$y$之间都没有新的请求对生成的概率。
\begin{align*}
& & g_s(y-\varepsilon) = g_s(y)(1-f(y)\varepsilon) \text{且} g_s(s) = 1\\
& \Rightarrow & g_s'(y) = \frac{g_s(y) - g_s(y-\varepsilon)}{\varepsilon} = g_s(y) * f(y)\\
& \Rightarrow & g_s'(y) = g_s(y)f(y) \Rightarrow g_s(y) = e^{F(y)-F(s)}
\end{align*}

现在需要计算 $OPT$ 和任何在线确定性算法 $ALG$ 产生的期望成本。需要注意的是,构造过程是状态化的,也就是说,对于状态值为 $s$ 的任意时间点后构造的序列的分布,只与 $s$ 有关。因此,假设某一时间点其状态值为 $s$,计算算法在这一点之后构造序列的期望成本,该值仅与状态值 $s$ 有关。首先讨论$OPT$产生的期望成本,这等于相应独立片段长度的期望值。
$T(s)$考虑两种情况:
 \renewcommand{\labelenumi}{(\arabic{enumi})}
\begin{enumerate}[align=left, labelsep=-0.5em, itemindent=3.6em, leftmargin=0em]
    \item 构造过程在之后没有生成请求对,独立片段增加长度$s$。这种情况的概率为$g_s(0)=e^{F(0)-F(s)}$。
    \item 构造过程在之后状态$y$生成请求对,独立片段增加长度$(s-y) +T(1-y)$。这种情况的概率密度为$g_s(y)=e^{F(y)-F(s)}$。
\end{enumerate}
因此可得递推式:
\begin{equation}
    T(s)=s e^{F(0)-F(s)}+\int_0^s[(s-y)+T(1-y)] e^{F(y)-F(s)} f(y)dy
\end{equation}
    
接下来计算$ALG$的期望成本,对任意 $ALG$ 参数化。由于$ALG$ 在任一时间点要么等待,要么空闲,并且在这一时间点后可以延迟等待,将延迟等待的时间参数化为 $m$\footnote{注意到 $m>s$ 的算法可以等价转化为 $m=s$ 的算法}。可获得 $ALG$ 对构造序列的成本计算\footnote{注意在$A(0,s)$中的$m^*$的值并不能确定},类似\ref{sec:3-lowerbound}节中的定义,将$ALG$等待状态的版本定义为$A(1,s,m)$,空闲状态的版本定义为$A(0,s)$,两者均可展开为积分递推式,$A(1,s,m)$考虑两种情况:
 \renewcommand{\labelenumi}{(\arabic{enumi})}
\begin{enumerate}[align=left, labelsep=-0.5em, itemindent=3.6em, leftmargin=0em]
    \item 构造过程在$m$时间内没有生成请求对,$ALG$会在$m$时间内部匹配,其成本为$1+2m$。此后$ALG$转换为空闲状态。这种情况的概率为$g_s(s-m)=e^{-m}$。
    \item 构造过程在$(0, m)$时间内生成请求对,其$ALG$外部匹配成本为$2(s-y)$,此后$ALG$转换为空闲状态。这种情况的概率密度为$f(y)e^{y-s}$。
\end{enumerate}
$A(0, s)$只会考虑构造算法生成某对请求点之后,算法便转换为等待状态,其概率密度为$e^{F(y)-F(x)}$,因此可得递推式:
\begin{align}
    \nonumber & A(1, s, m)=[(1+2m)+A(0, s-m)]e^{-m}+\int_{s-m}^s[2(s-y)+A(0, 1-y)]f(y)e^{y-s}dy \\
    &A(0, s)=\int_0^s A(1, 1-y, m^*) e^{F(y)-F(s)}f(y)dy 
\end{align}

此时目标是 $\max_{f(x)}\min_{m}{A(1, 1, m)/T(1)}$,即找到一种分布,能够攻击所有确定性算法,并找到其中成本比最大的一种分布。这类似于博弈论中扮演敌手,设计一个分布去攻击所有的确定性策略。因此可假设最优分布满足\emph{混合策略纳什均衡},即所有确定性算法都具有相同的成本。

假设$ALG$延迟等待任意值在$\tlr$上的期望成本都相同,则假设以下等式成立:
\begin{align}
     \label{eq:assumption}\forall m, ~~~~  A(1, s, m) & = a(s) \Rightarrow \frac{d}{dm}a(s) = 0\\ 
    A(0, s) & = b(s) 
\end{align}
让$A(1, s, m)$对$m$求导可得:
\begin{align}
   \nonumber \frac{d}{dm}A(1,s,m) = & 2 m e^{- m} f(s-m) - 2 m e^{- m} - b{\left(- m + s \right)} e^{- m} \\
   & + b{\left(m - s + 1 \right)} e^{- m} f(s-m) - e^{- m} b'(s-m) + e^{- m}
\end{align}

重整化,令$s-m = t$, 联立$a'(s)=0$的假设可得方程:
\begin{equation} \label{eq:lowerbound_1}
    2mf(t) - 2m - b(t) + b(1-t)f(t) - b'(t) + 1 = 0 
\end{equation}

同时,让$A(0, s)$对$s$求导可得:
\begin{align}
    \nonumber \frac{d}{dm}A(0,s) & = \\ 
    \nonumber  b'(s) & = a(1-s)f(s) - f(s)\int\limits_{0}^{s} a{\left(1 - y \right)} e^{- F{\left(s \right)} + F{\left(y \right)}} f(y)\, dy\\
    b'(s) & = a(1-s)f(s) - f(s)b(s) \label{eq:lowerbound_2}
\end{align}
同时,可得:
\begin{equation} \label{eq:lowerbound_3}
    A(1,s,0) = a(s) = 1 + b(s)     
\end{equation}

联立方程~\ref{eq:lowerbound_1}、~\ref{eq:lowerbound_2}和\ref{eq:lowerbound_3}可得:
\begin{align*}
     & & b'(t) & = 2mf(t)-2m-b(t)+b(1-t)f(t) + 1 \\
     & &  & = 1+b(1-t)f(t) - b(t)f(t) \\
    & \Rightarrow & f(t)(2m+b(t)) & = 2m+b(t) \\
    & \Rightarrow & f(t) & = 1
\end{align*}

因此可得$f(s)=1$。
将其代入$T(s), A(1,s,m)$可推出:
\begin{align}
    T(s) & = -s^2 + 2s \\
    A(1, s, m) & = -s^2 + 2s + 1
\end{align}

取$s=1$,可以得到目标值$T(1)=1,A(1,1,m)=2$,与上一节归纳证明的结果一致,并且发现算法$A(1, 1, m)$的值与$m$无关,满足假设的等式\ref{eq:assumption}。
\section{本章小结}